\section{Referencial Teórico}
\subsection{Compiladores}
Segundo \citeonline{new-dragon-pt}, \compilador{} é um programa que traduz um
programa-fonte para um programa-objeto. Se o programa-objeto for executável,
então ele estará num formato que um computador possa executá-lo. Dessa forma, um compilador recebe
como entrada um arquivo contendo um programa escrito em uma linguagem previamente
determinada (programa-fonte) e produz como saída um programa objeto semanticamente equivalente
ao programa recebido como entrada. Dessa forma, podemos dizer que um \compilador é,
também, um tradutor. Adicionalmente o \compilador tem como tarefa reportar os
erros encontrados durante o processo de tradução.

O \compilador pode ser dividido em alguns módulos para efetuar o processo de
tradução. A lista abaixo foi proposta por \citeonline{louden97-pt}:

\begin{itemize}
	\item Analisador Léxico;
	\item Analisador Sintático;
	\item Analisador Semântico;
	\item Gerador de Código.
\end{itemize}

O \emph{Analisador Léxico} é o responsável por agrupar os caracteres, contidos
no arquivo do programa-fonte, em unidades significativas, chamadas
\emph{tokens}, e encaminhá-las para o \emph{Analisador Sintático}.

Por sua vez, o \emph{Analisador Sintático} verifica se o fluxo de tokens
recebidos pelo Analisador Léxico é válido para a gramática (ou linguagem) que
foi definida. Usualmente o Analisador Sintático produz uma estrutura
(tipicamente uma do tipo árvore) que representa o programa fonte
\cite{new-dragon-pt}.

O \emph{Analisador Semântico} recebe a estrutura produzida pelo Analisador
Sintático e, principalmente, verifica se as operações são coerentes para os
tipos de dados utilizados e faz a inferência dos tipos de dados. Ao final do
processo temos uma \emph{Árvore Anotada}. As notas (informações de inferência,
por exemplo) podem ser incluídas diretamente na estrutura recebida do
Analisador Sintático, ou numa estrutura auxiliar como uma \emph{Tabela de
Símbolos}.

Estando a Árvore Anotada disponível para o \emph{Gerador de Código}, este
executa a tradução das estruturas recebidas para o programa-objeto. Nessa fase
da compilação podem ser incluídas otimizações para que o programa traduzido
execute de forma mais eficiente.

Podem haver outras fases intermediárias durante o processo, como, por exemplo,
as fases de Otimização de Código, dependentes ou não da máquina-alvo.

Nas próximas Seções, discutiremos mais profundamente cada uma dessas etapas
do processo de compilação.

\subsection{Analise Léxica}

O processo de Analise Léxica consiste em agrupar os caracteres do arquivo de
entrada em unidades numa estrutura chamada \token. Um token também é chamado
de \emph{lexema}.

Segundo \citeonline{dict-aurelio}, ``lexema é o elemento que encerra o
significado da palavra''. Ou seja, é o menor conjunto de caracteres
representativos para uma gramática de uma linguagem. Dessa forma, o Analisador
Léxico remove a responsabilidade de representar os tokens do Analisador
Sintático, simplificando sua implementação.

Segundo \citeonline{new-dragon-pt}, os tokens são definidos como segue:

\begin{citacao}{4cm}{0cm}\footnotesize \emph
	Um token consiste em dois componentes, um nome de token e um balor de
	atributo. Os nomes de token são símbolos abstratos usados pelo analisador para
	fazer o reconhecimento sintático. Frequentemente, chamamos esses nomes de
	token de \emph{terminais}, uma vez que eles aparecem como \emph{símbolos
	terminais} na gramática para uma linguagem de programação. O valor do
	atributo, se houver, é um apontador para a tabela de símbolos que contém
	informações adicionais sobre o token. (\ldots).
\end{citacao}

Ainda segundo \citeonline{new-dragon-pt}, o Analisador Léxico possui algumas
atribuições adicionais, como por exemplo, remover espaços em branco e
comentários, efetuar contagem de linhas correlacionando um erro com o número
da linha em que este foi encontrado.

Tipicamente, o Analisador Léxico não gera o todo o fluxo de tokens de uma vez.
Ao invez disso, a demanda de análise dos tokens fica sob a responsabilidade do
Analisador Sintático, que recebe os tokens ativando uma função disponibilizada
pelo Analisador Léxico \cite{louden97-pt}.



\subsection{Análise Sintática}

A Análise Sintática define a forma com que um programa é estruturado. Essa
estrutura é dada por um conjunto de \emph{regras gramaticais} descritas em uma
\emph{Gramática Livre de Contexto},(verificar Seção \ref{sec:context_free_grammar}).

Segundo \citeonline{new-dragon-pt}:
\begin{citacao}{4cm}{0cm}\footnotesize \emph
	Existem três estratégias gerais de análise sintática para o processamento
	de gramáticas: universal, descendente e ascendente. Os métodos de análise
	baseados na estratégia universal (\dots) podem analisar qualquer
	gramática, (\dots) no entanto são muito ineficientes para serem utilizados
	em compiladores de produção.

	Os métodos geralmente usados em compiladores são baseados nas estratégias
	descendentes ou ascendentes. Conforme sugerido por seus nomes, os métodos
	de análise descendentes constroem as árvores de derivação de cima (raiz)
	para baixo (folhas), enquanto os métodos ascendentes fazem a análise no
	sentido inverso, começam nas folhas e avançam até a raiz construindo a
	árvore. Em ambas as estratégias, a entrada do analisador sintático é
	consumida da esquerda para a direita, um símbolo de cada vez.
\end{citacao}

Para maiores referências sobre analisadores descendentes, consulte
\citeonline{louden97-pt}, \citeonline{parr07}, \citeonline{jacobs85}.

\subsubsection{Gramáticas Livres de Contexto}
\label{sec:context_free_grammar}

\emph{Gramática} é essencialmente um conjunto de Regras de Produção (ou
Re-escrita). Essas regras são, usualmente, descritas utilizando uma notação
chamada \textbf{Forma de Backus-Naur}, ou \textbf{BNF} \cite{louden97-pt}.

Um exemplo abstrato de regra de produção é demonstrado abaixo:

\[
A \rightarrow \alpha
\]

Esta expressão indica que o não-terminal \(A\) será substituído pela sequência
de terminais e/ou não-terminais representada por \(\alpha\). Um \emph{terminal},
normalmente, é um \emph{token} oriundo do analisador léxico.

Um exemplo mais concreto é demonstrado abaixo:

\[
expr \rightarrow expr + expr | \textbf{numero}
\]

Esta regra indica que uma expressão é composta de uma expressão seguida de um
sinal de + seguida de outra expressão, ou de um número. O nome da regra é dado
pela parte que está a esquerda da seta, seu corpo é dado pelo que está a
direita. O sinal \(|\) indica uma escolha de alternativas no corpo da
produção. Percebemos, também, que uma regra gramatical pode ter uma
definição recursiva.

Segundo \citeonline{louden97-pt}, podemos definir uma \textbf{gramática livre de
contexto} mais formalmente conforme segue:
\begin{enumerate}
	\item Um conjunto \(T\) de \textbf{terminais}.
	\item Um conjunto \(N\) de \textbf{não-terminais} (disjunto de \(T\)).
	\item Um conjunto \(P\) de \textbf{produções} na forma \(A \rightarrow \alpha\)
				em que \(A\) é um elemento de \(N\) e \(\alpha\) é um elemento de
				\((T \cup N)^*\) (uma sequência de terminais e não-terminais que
				pode ser vazia).
	\item Um \textbf{símbolo inicial} \(S\) do conjunto \(N\).
\end{enumerate}

Dessa forma, o processo de reconhecimento da linguagem inicia-se derivando o
símbolo inicial da gramática, substituindo repetidamente um não-terminal pelo
corpo desse não terminal \cite{new-dragon-pt}.

Assim, uma \textbf{gramática livre de contexto} é uma \emph{gramática}
conforme definido a cima, e é livre de contexto pois a parte a esquerda de uma
regra de produção pode ser substituída pelo seu corpo em qualquer ponto,
independentemente de onde ocorra a parte esquerda da regra \cite{louden97-pt}.

Em contrapartida, uma produção em uma gramática sensível a contexto é demonstrada
abaixo:
\[
	\gamma{}A{}\beta \rightarrow \gamma\alpha\beta
\]

Nesta regra, \(A\) pode ser substituído por \(\alpha\), somente se
\(A\) estiver entre os terminais \(\gamma\) e \(\beta\).

\subsubsection{Análise Sintática Ascendente}
\label{sec:asc_syntax_analisys}


\subsubsection{Geradores de Analisadores Sintáticos Ascendentes}
\label{sec:yacc}

\subsection{Tabela de Símbolos}
\label{sec:symtab}

Segundo \citeonline{new-dragon-pt}, ``\emph{Tabelas de Símbolos} são
estruturas de dados utilizadas pelos compiladores para conter informações
sobre as construções do programa-fonte''. Essas informações são coletadas
durante as fases de análise (Léxica e Sintática) e utilizadas durante
a fase de geração do programa-objeto (também conhecida como fase de
\emph{síntese}).

As entradas na tabela de símbolos contém informações sobre
identificadores; nome ou seu lexema, posição de memória, seu tipo, entre
outras informações que o implementador julgar necessárias.

As principais operações sobre Tabelas de Símbolos são \emph{inserir},
\emph{consultar} e \emph{remover} uma entrada. Dessa forma, precisamos de uma
estrutura de dados que permita executar essas operações eficientemente.
Foi escolhida a estrutura de \emph{Tabela de Hash Encadeada} para sua
implementação (verificar Seção \ref{sec:hashing}).

Tabelas de símbolos, também, são comumente utilizadas para manter
informações de escopo dos identificadores. Como neste projeto teremos
apenas um escopo global, não discutiremos esse tema, entretanto, mais
referências podem ser encontradas em \citeonline{new-dragon-pt} e
\citeonline{louden97-pt}.


\subsubsection{Hashing (Transformação de Chave)}
\label{sec:hashing}

\emph{Hashing} é um método de pesquisa que utiliza uma função de
transformação da chave de pesquisa para calcular o endereço em que a
entrada será armazenada.

\begin{figure}
	\begin{center}
		\includegraphics[scale=0.6]{hashtable.png}
	\end{center}
	\caption{Tabela de Hashes}
	\label{fig:hashtable}
\end{figure}

Como podemos observar na Figura \ref{fig:hashtable}, temos as chaves $6, 11,
16, 19, 22, 27$ inseridas na tabela. Também é possível perceber que a
chave com valor $16$ está inserida no endereço $0$ da tabela. Para efetuar
o mapeamento entre as chaves e os endereços é necessário a utilização de
uma função de \emph{hashing} (ou função de \emph{transformação}).

Uma \textbf{função de hashing} deve mapear uma chave, o nome de um
identificador, por exemplo, em inteiros dentro do intervalo $[0..M-1]$ em
que $M$ é o tamanho da tabela. Considerando que as transformações sobre
as chaves são aritméticas, o primeiro passo é transformar as chaves
não-numéricas em números. Para isso, podemos utilizar, por exemplo, o valor
inteiro conforme a \emph{Tabela ASCII} \cite{ziviani}.

\begin{equation} \label{eq:hash}
h(K) = K \text{mod} \, M
\end{equation}

A Equação \ref{eq:hash} demonstra uma das formas de calcular a função
\emph{hash} de uma chave $K$. Nela calculamos o resto da divisão de $K$
pelo tamanho $M$ do arranjo que armazenará a tabela.

\begin{equation} \label{eq:k}
K = \sum_{i=0}^{n-1}\text{chave}[i] \times i
\end{equation}

$K$ é definido conforme a Equação \ref{eq:k}. $i$ é o índice do caractere na
cadeia \emph{chave} e $n$ é o tamanho da cadeia. O produto por $i$ é utilizado
para evitar \emph{hashes} iguais quando tratamos de \emph{anagramas}.

Nesse processo há grande possibilidade de duas chaves possuírem \emph{hashes}
iguais, isto é denominado \emph{colisão} de hashes. Uma das formas possíveis
de resolução de \emph{colisões} é utilizar uma \emph{Lista Encadeada}. Dessa
forma, todas as chaves conflitantes são encadeadas em uma lista linear
\cite{ziviani}. Uma demonstração dessa representação é de dada na Figura
\ref{fig:hashtable}.

Outras formas de resolução de \emph{colisões} e outras implementações de
hashes (como Hashing Perfeito) podem ser encontrados em \citeonline{knuth73} e
\citeonline{ziviani}.


\subsection{Geração de Código}
\emph{Geração de Código} é o processo de utilizar todas as informações
geradas durante as fases de \emph{análise} (Léxica, Sintática etc) para
gerar o programa-objeto. Conforme a arquitetura do compilador, é possível
incluir outras etapas intermediárias, conhecidas como \emph{Representações
Intermediárias} (RI), que visam possibilitar otimizações no programa-objeto
gerado \cite{louden97-pt}.

Uma forma possível de RI é conhecida com \emph{código-de-três-endereços}. Este
formato é conhecido desta forma pois possui a seguinte forma de instruções $x
= y \textbf{op} z$, ou seja, do lado direito da atribuição possui apenas um
operador binário, seus operandos e do lado esquerdo a variável que armazena o
resultado da operação. Variações são permitidas para representar, por exemplo,
o sinal de menos unário $x = -y$.

\begin{lstlisting}[label=lst:three_addresses,caption=Código de 3 Endereços]
t1 = c * d
t2 = a + b
t3 = t1 + t2
x = t3
\end{lstlisting}

A Listagem \ref{lst:three_addresses} demonstra um exemplo do
código-de-três-endereços para a expressão $x=a+b+c*d$. As variáveis
$\text{t}i$ para $i \in \{1,2,3\}$ representam variáveis temporárias criadas pelo próprio
compilador.

Para este projeto, não são geradas RIs, apenas os programas-objeto em
\emph{Linguagem C}, que posteriormente podem ser compiladas por um compilador
C, como o \emph{gcc}, gerando um programa executável, e em \emph{Linguagem
DOT} possibilitando a geração de uma representação gráfica do programa.
Mais referências sobre RIs e códigos-de-três-endereços são encontradas em
\citeonline{new-dragon-pt} e \citeonline{louden97-pt}.

Conforme exposto, este projeto de compilador atua como um tradutor entre
linguagens. Uma abordagem semelhante foi utilizada na implementação inicial da
\emph{Linguagem C++}. Este compilador traduzia programas C++ para programas C
para que pudessem, posteriormente, ser compilados por um compilador C
disponível. Assim, podemos considerar o processo de compilação como um
processo de tradução de uma linguagem de nível mais alto para uma outra
linguagem de nível mais baixo, repetindo o  processo até que seja produzido um
programa executável na máquina-alvo \cite{new-dragon-pt}.

A Geração de Código também consiste num processo de linearização das
estruturas de árvores disponibilizadas pelas fases anteriores, transformando,
por exemplo, uma árvore sintática em um programa C, em que as instruções são
escritas linearmente em um arquivo.

A Listagem \ref{lst:code_gen} demonstra uma possível implementação, em
pseudo-código, de função geradora de código, tendo como base uma árvore
sintática em que cada nó possui até dois filhos. Notamos que a função pode
vistar a árvore em pré-ordem, em ordem e pós-ordem.

\begin{lstlisting}[label=lst:code_gen,caption=Exemplo Gerador de Código]
funcao geraCodigo (no_arvore T)
inicio
	gerar_codigo_preparatorio(T)
	gerar_codigo(T)
	gerar_codigo_preparatorio_filho_esquerda(T->filho_esquerda)
	gerar_codigo_filho_esquerda(T->filho_esquerda)
	gerar_codigo_preparatorio_filho_direita(T->filho_direita)
	gerar_codigo_filho_direita(T->filho_direita)
	gerar_codigo_final(T)
fim
\end{lstlisting}

Com pequenas alterações no código da Listagem \ref{lst:code_gen}, podemos
incluir mais filhos aos nós filhos à árvore $T$, bem como, representar a
construção de quase todas as contruções necessárias para produzir o
programa-objeto.


\subsubsection{Linguagem DOT}
\label{sec:impl_gen_dot}

Segundo \citeonline{EGKNW03}:

\begin{citacao}{4cm}{0cm}
Graphviz is a collection of software for viewing and manipulating abstract
graphs. It provides graph visualization for tools and web sites in domains
such as software engineering, networking, databases, knowledge representation,
and bio-informatics
\end{citacao}

Um dos softwares dessa coleção é o compilador \emph{dot}. Segundo
\citeonline{gansner09}:

\begin{citacao}{4cm}{0cm}
\textbf{dot} draws directed graphs. It reads attributed graph text files and
writes drawings, either as graph files or in a graphics format such as GIF,
PNG, SVG, PDF, or PostScript.
\end{citacao}

\textbf{dot} aceita como entrada um arquivo de texto expresso na Linguagem
DOT (verificar \url{http://graphviz.org/content/dot-language}). Essa
linguagem define três tipos principais de objetos: grafos, nós e arestas.
O grafo principal (mais externo) pode ser direcionado (\emph{digraph} {--}
\emph{directed graph}, ou seja, grafo direcionado), ou não-direcionado. Em um
grafo principal, é possivel termos um sub-grafo (\emph{subgraph}) que permite
a definições de nós e arestas \cite{gansner09}.

Um nó é criado quando o seu nome aparece pela primeira vez no arquivo. As
arestas são criadas quando dois nós são ligados pelo operador de aresta {->}.

Na Listagem \ref{lst:dot_example} temos o exemplo de um grafo escrito em
DOT que após sua compilação com o comando
$$
\text{dot} \quad \text{-Tpng} \quad \text{exemplo\_dot.gv} \quad \text{>}
\quad \text{exemplo\_dot.png}
$$
gerará a representação gráfica demonstrada na Figura \ref{fig:dot_example}.

\lstinputlisting[label=lst:dot_example, caption=Exemplo de Grafo Expresso em DOT]{src_files/dot_example.gv}

\begin{figure}
	\begin{center}
		\includegraphics[scale=0.4]{dot_example}
	\end{center}
	\caption{Exemplo Grafo Gerado pelo dot}
	\label{fig:dot_example}
\end{figure}


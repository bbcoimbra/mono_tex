\begin{abstract} %%%\section*{Resumo}

Este trabalho tem como principal objetivo descrever a implementação de um
compilador capaz de gerar uma representação gráfica da lógica do programa.
Mais precisamente, este compilador gera duas representações, semanticamente
equivalentes, do programa-fonte: uma em linguagem C, que pode ser compilada
por um compilador C padrão, e outra em linguagem DOT que, ao ser compilada,
gera uma representação gráfica da lógica do programa. Espera-se que esta
representação gráfica seja uma ferramenta que facilite a aprendizagem
de programação, uma vez que ela torna explícito fluxo de execução do
programa para os novatos nesta área.

\vspace{5cm}

\textbf{Palavras-chave: compiladores, Linguagem C, DOT, Lex, Yacc}
\end{abstract}

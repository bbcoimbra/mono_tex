\begin{abstract} %%%\section*{Resumo}

O presente trabalho tem como objetivo demonstrar a implementação
de um pequeno compilador. São demonstrados os módulos que o
compõe e como resultado final são geradas duas representações,
semanticamente equivalentes, do programa-fonte, uma em Linguagem C,
que pode ser compilada por um compilador C e executada; e outra em
linguagem DOT, que ao ser copilada gera uma representação gráfica
do programa. Adicionalmente, são apresentadas ferramentas que
auxiliam na implementação do compilador, bem como, estruturas de
dados adicionais necessárias para o processo de compilação.

\vspace{5cm}

\textbf{Palavras-chave: compiladores, Linguagem C, DOT, Lex, Yacc}
\end{abstract}

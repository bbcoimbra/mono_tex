\subsection{Analise Léxica}

O processo de Analise Léxica consiste em agrupar os caracteres do arquivo de
entrada em unidades numa estrutura chamada \token. Um token também é chamado
de \emph{lexema}.

Segundo \citeonline{dict-aurelio}, ``lexema é o elemento que encerra o
significado da palavra''. Ou seja, é o menor conjunto de caracteres
representativos para uma gramática de uma linguagem. Dessa forma, o Analisador
Léxico remove a responsabilidade de representar os tokens do Analisador
Sintático, simplificando sua implementação.

Segundo \citeonline{new-dragon-pt}, os tokens são definidos como segue:

\begin{citacao}{4cm}{0cm}\footnotesize \emph
	Um token consiste em dois componentes, um nome de token e um balor de
	atributo. Os nomes de token são símbolos abstratos usados pelo analisador para
	fazer o reconhecimento sintático. Frequentemente, chamamos esses nomes de
	token de \emph{terminais}, uma vez que eles aparecem como \emph{símbolos
	terminais} na gramática para uma linguagem de programação. O valor do
	atributo, se houver, é um apontador para a tabela de símbolos que contém
	informações adicionais sobre o token. (\ldots).
\end{citacao}

Ainda segundo \citeonline{new-dragon-pt}, o Analisador Léxico possui algumas
atribuições adicionais, como por exemplo, remover espaços em branco e
comentários, efetuar contagem de linhas correlacionando um erro com o número
da linha em que este foi encontrado.

Tipicamente, o Analisador Léxico não gera o todo o fluxo de tokens de uma vez.
Ao invez disso, a demanda de análise dos tokens fica sob a responsabilidade do
Analisador Sintático, que recebe os tokens ativando uma função disponibilizada
pelo Analisador Léxico \cite{louden97-pt}.

O reconhecimento dos tokens é feito utilizando duas técnicas principais:
\emph{Expressões Regulares e Autômatos Finitos}.

\subsubsection{Expressões Regulares}

Segundo \citeonline{regex-jargas}:
\begin{citacao}{4cm}{0cm} \footnotesize \emph
	Resumidamente, uma expressão regular é um método formal de se especificar um padrão de texto.

	Mais detalhadamente, é uma composição de símbolos, caracteres com funções
	especiais, que, agrupados entre si e com caracteres literais, formam uma
	seqüência, uma expressão. Essa expressão é interpretada como uma regra, que
	indicará sucesso se uma entrada de dados qualquer casar com essa regra, ou
	seja, obedecer exatamente a todas as suas condições
\end{citacao}

As expressões regulares são uma importante notação para especificar os padrões
dos lexemas. Mesmo não podendo especificar todos os padrões possíveis elas são
muito eficientes para o propósito de especificar os tokens que necessitamos.

\textbf{Definições} (segundo \citeonline{new-dragon-pt}):
\begin{description}
	\item[Alfabeto] é qualquer conjunto finito de símbolos. Temos como exemplos
		de símbolos as letras, dígitos etc. O conjunto $\{0, 1\}$ representa o
		\emph{alfabeto binário}.
	\item[Cadeia] em um alfabeto é uma sequência finita de símbolos retirados de
		alfabeto. Normalmente, o tamanho da cadeia $s$ é dado por $|s|$. Por
		exemplo ``compilador'' é uma cadeia de tamanho 10. A cadeia vazia,
		indicada por $\epsilon$, tem tamanho zero.
	\item[Linguagem] é qualquer conjunto contável de cadeias de algum alfabeto.
\end{description}

\textbf{Expressão Regular Básica} são, simplesmente, os caracteres separados do
alfabeto que casam com eles mesmos. Por exemplo, dado que definimos o conjunto
dos caracteres ASCII como nosso alfabeto, a expressão regular $/a/$ casa com o
caractere \textbf{a}.

Há três operações básicas utilizando Expressões Regulares conforme descritas
abaixo \cite{louden97-pt}:

\begin{description}
	\item[Escolha Entre Alternativas] Dado que $r$ e $s$ são expressões
		regulares, então $r | s$ é uma expressão regular que case com a expressão
		$r$ ou com a expressão $s$. Exemplo: dado que $r$ seja a expressão regular
		$/a/$ e $s$ a expressão regular $/b/$, $r | s$ casa com o caractere
		\textbf{a} ou o caractere \textbf{b}.
	\item[Concatenação] A concatenação de duas expressões regulares $r$ e $s$ é
		da pela expressão $rs$ e casa com qualquer cadeia que case com a expressão
		regular $r$ seguida pela expressão regular $s$. Exemplo: dado que $r$ seja
		a expressão regular $/ca/$ e $s$ a expressão regular $/sa/$, $rs$ casa
		com a cadeia ``casa''.
	\item[Repetição] Também conhecida como fecho de Kleene, é denotada por $r*$,
		em que $r$ é uma expressão regular. A expressão regular $r*$ representa o
		conjunto de cadeias obtidas pela concatenação de zero ou mais expressões
		regulares $r$. Exemplo: dada a expressão regular $/a*/$, esta expressão
		casa com as cadeias \textbf{$\epsilon$}, \textbf{a}, \textbf{aa},
		\textbf{aaa}, \textbf{aaaa}, \dots.
\end{description}

Para simplificar a notação das expressões regulares é comum associar nomes às
expressões regulares longas. Uma Expressão Regular nomeada é chamadas de
\textbf{definição regular}.

Além das operações descritas, a norma \citeonline{POSIX.1} define operações
adicionais chamadas Expressões Regulares Extendidas:

\begin{description}
	\item[Uma ou Mais Repetições] A repetição de de uma ou mais vezes da
		expressão regular $r$ é dada por $r+$, eliminando o casamento da expressão
		vazia (\(\epsilon\)). O mesmo resultado poderia ser obtido com a expressão
		$rr*$, mas esta é uma situação tão frequente que foi simplificada e
		padronizada \cite{louden97-pt}.
	\item[Qualquer Caractere] Um ``.'' (ponto) é utilisado para efetuar o
		casamento com qualquer caractere, exceto um caractere nulo.
\end{description}

Há outras operações que envolvem expressões regulares. Para mais informações
colsulte \cite{regex-jargas}, \cite{POSIX.1} e \cite{louden97-pt}.

\subsubsection{Autômatos Finitos}

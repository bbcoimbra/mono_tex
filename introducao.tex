\section{Introdução}
Este trabalho tem como principal objetivo descrever a implementação de um
compilador capaz de gerar uma representação gráfica da lógica do programa.
Mais precisamente, este compilador gera duas representações, semanticamente
equivalentes, do programa-fonte: uma em linguagem C, que pode ser compilada
por um compilador C padrão, e outra em linguagem DOT que, ao ser compilada,
gera uma representação gráfica da lógica do programa. Espera-se que esta
representação gráfica seja uma ferramenta que facilite a aprendizagem
de programação, uma vez que ela torna explícito fluxo de execução do
programa para os novatos nesta área.

Serão apresentadas técnicas e exemplos de análise léxica e sintática,
geração de código objeto e algumas estruturas de dados necessárias para a
implementação. Também serão demonstrados neste trabalho a utilização de
ferramentas de auxílio ao desenvolvimento de compiladores, como geradores de
analisadores léxicos e sintáticos.

O trabalho foi estruturado de forma que o leitor faça uma leitura linear,
sem que haja saltos entre as seções. Na primeira seção deste trabalho
(Referencial Teórico) são apresentados, de forma sucinta, os conceitos
fundamentais sobre os tema.

Na seção seguinte (Implementação) é discutida a implementação do
compilador propriamente dito, quais foram as técnicas, ferramentas,
algoritmos e estruturas de dados utilizadas. A última seção (Conclusão)
apresenta os resultados obtidos, bem como as limitações do projeto e
sugestões de melhoria.

No Apêndice encontra-se a listagem completa dos programas-fonte.
O projeto completo também pode ser encontrado em
\url{http://github.com/bbcoimbra/compiler}, mesmo local em que serão
incluídas as correções e melhorias.
